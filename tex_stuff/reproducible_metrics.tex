\documentclass[12pt]{article}
%search and replace this in notepad++: 	{\\bf(.+?)}
\usepackage[labelfont=bf,textfont=bf]{caption}
\usepackage{graphicx}
\usepackage{epstopdf}
\usepackage{natbib, microtype, color, amsmath, graphics, dcolumn, booktabs, multicol}
\usepackage{multirow} 
\usepackage{amsfonts,dsfont}
\usepackage{setspace, lscape, longtable, rotating}
\usepackage{geometry}
\geometry{margin=1in}
\usepackage[normalem]{ulem}
\usepackage{appendix}
\usepackage{mathtools}
\usepackage{eurosym}
\usepackage{verbatim}
\DeclarePairedDelimiter{\ceil}{\lceil}{\rceil}
%\usepackage[nomarkers,nolists]{endfloat} %note, needs endfloat.cfg copied from efxmpl.cfg to work properly
%\renewcommand{\efloatseparator}{\mbox{}} %for endfloat
\newcommand{\putat}[3]{\begin{picture}(0,0)(0,0)\put(#1,#2){#3}\end{picture}}
\newcommand{\possessivecite}[1]{\citeauthor{#1}'s \citeyearpar{#1}}
%\usepackage{gaggl}
\widowpenalty=10000 \clubpenalty=10000 %keeps pdf open properties you like
\usepackage[bookmarks=false]{hyperref} 
\hypersetup{
	pdfauthor = {Nadav Tadelis}, 
    pdftitle = {Reproducible Econometrics}, 
    pdfsubject = {A reproducible approach to analyzing the effect of studying on grades},
	pdfborder =0 0 0, bookmarksopen=false, colorlinks=false,
%	pdfkeywords = {Keyword1, Keyword2, ...}, pdfcreator = {LaTeX with hyperref package}, pdfproducer = {dvips + ps2pdf}}
}
\usepackage{subcaption}
\newenvironment{changemargin}[2]{%
  \begin{list}{}{%
    \setlength{\topsep}{0pt}%
    \setlength{\leftmargin}{#1}%
    \setlength{\rightmargin}{#2}%
    \setlength{\listparindent}{\parindent}%
    \setlength{\itemindent}{\parindent}%
    \setlength{\parsep}{\parskip}%
  }%
  \item[]}{\end{list}} 
\linespread{1.3} 
%\doublespacing
\def\sym#1{\ifmmode^{#1}\else\(^{#1}\)\fi}

\newtheorem{claim}{Claim}
\newtheorem{definition}{Definition}

%ADDED FOR TABLE ROW HEIGHT
\usepackage{array}
\newcolumntype{M}[1]{>{\centering\arraybackslash}m{#1}}
\newcolumntype{N}{@{}m{0pt}@{}}


\begin{document}

\title{ROUGH FRAMEWORK FOR PAPER (using old material as filler for now)}

\date{First draft: FILL\\ Current Draft: FILL}

\author{ \\ Nadav Tadelis}

\pagenumbering{gobble} 

\maketitle

\hskip 80pt 


\begin{abstract}
\footnotesize{
\noindent ABSTRACT HERE \vfill}
\end{abstract}

\clearpage

\pagenumbering{arabic} 

%***********************

\section{Introduction}
\label{sec_intro}
Intro

%***********************

\section{Defining the Model and Assumptions}
\label{sec_background}

Let $Z_i$ represent the treatment status of individual $i$ where $Z_i=0$ when an individual is not treated, and $Z_i=1$ when an individual is treated. When using a proxy model, assignment of the treatment is the same for all individuals.
\begin{definition} \label{D1} (Treatment) Under a proxy, the treatment status of one individual is analogous to the treatment status of the population, i.e.,
\begin{equation*}
%\tag{C1} 
Z_i=Z. 
\end{equation*}
\end{definition}

\subsection{Comparison to the Instrumental Variable model}
\label{sec_background_iv}

\begin{equation*}
\label{eq_sutva}
\begin{split}
\text{The }&\text{SUTVA (Rubin, 1990):}\\
& a. \text{ If } Z_{i} = Z_{i}^{\prime} \text{ then } R_{i}(\mathbf{Z}) = R_{i}(\mathbf{Z}^{\prime}).\\
& b. \text{ If } Z_{i} = Z_{i}^{\prime} \text{ and } R_i = R_{i}^{\prime}, \text{ then } S_{i}(\mathbf{Z}, \mathbf{R})=S_{i}(\mathbf{Z}^{\prime}, \mathbf{R}^{\prime}).
\end{split}
\end{equation*}
Part $a$ of the SUTVA states that an individual's value $R_i$ is only dependent on her own treatment status $Z_i$; i.e. the treatment status of other individuals $Z_j, j \neq i$, does not affect $R_i$. Part $b$ requires that the potential outcomes $S_{i}(\mathbf{Z}, \mathbf{R})$ of $i$ are independent of the treatment status's ($Z_j$) and risks ($R_{j}(\mathbf{Z})$) of other individuals. Clearly, from Definition 1, part $a$ will always be satisfied under a proxy model.
\begin{claim} (Assumptions) When applied to a proxy model, the original SUTVA from Rubin's Causal model simplifies to:
\begin{equation*}
%\tag{C1} 
\text{ If } Z_{i} = Z_{i}^{\prime} \text{ and } R_i = R_{i}^{\prime}, \text{ then } S_{i}(\mathbf{Z}, \mathbf{R})=S_{i}(\mathbf{Z}^{\prime}, \mathbf{R}^{\prime}).
\end{equation*}
\end{claim}
%********************************

%********************************

\newpage
\section*{References}
\begin{footnotesize}

\begin{description}

\item Angrist, J.D., Imbens, G.W., and Rubin, D.B. (1996), ``Identification of Causal Effects Using Instrumental Variables,'' \emph{Journal of the American Statistical Association}, 91:444-455. 

\item Angrist, J.D., and Lavy, V. (1999), ``Using Maimonides' Rule to Estimate the Effect of Class Size on Scholastic Achievement,'' \emph{The Quarterly Journal of Economics}, 114(2):533-575.


\end{description}
\end{footnotesize}\newpage

%******************************** bibliography test ********************************
citation test \cite{latexcompanion}

\bibliographystyle{abbrv}
\bibliography{tex_stuff/RM_bibliography}
%********************************

\end{document}




















